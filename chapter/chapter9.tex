\chapter{Distance measures for quantum information}
\Textbf{9.1}
\begin{align*}
	D((1,0), (1/2,1/2)) &= \frac{1}{2} \left( |1 - 1/2| + |0 - 1/2| \right)\\
		&= \frac{1}{2} \left(\frac{1}{2} +  \frac{1}{2} \right)\\
		&= \frac{1}{2}
\end{align*}

\begin{align*}
	D\left( (1/2,1/3,1/6), (3/4, 1/8, 1/8)  \right) &= \frac{1}{2} \left( |1/2 - 3/4| + |1/3 - 1/8| + |1/6 - 1/8| \right)\\
		&= \frac{1}{2} \left( 1/4 + 5 / 24 + 1/24 \right)\\
		&= \frac{1}{4}
\end{align*}



\Textbf{9.2}
\begin{align*}
	D\left((p,1-p), (q, 1-q)\right) &= \frac{1}{2} \left( |p-q| + |(1-p) - (1-q)| \right)\\
		&= \frac{1}{2} \left( |p-q| + |-p + q| \right)\\
		&= |p-q|
\end{align*}


\Textbf{9.3}
\begin{align*}
	F((1,0), (1/2,1/2)) = \sqrt{1 \cdot 1/2} + \sqrt{0 \cdot 1/2} = \frac{1}{\sqrt{2}}
\end{align*}

\begin{align*}
	F\left( (1/2,1/3,1/6), (3/4, 1/8, 1/8)  \right) &= \sqrt{1/2 \cdot 3/4} + \sqrt{1/3 \cdot 1/8} + \sqrt{1/6 \cdot 1/8}\\
		&= \frac{4 \sqrt{6} + \sqrt{3}}{12}
\end{align*}



\Textbf{9.4}

Define $r_x = p_x - q_x$. Let $U$ be the whole index set.
\begin{align*}
	\max_S |p(S) - q(S)| &= \max_S \left|\sum_{x \in S} p_x - \sum_{x \in S} q_x \right|\\
		&= \max_S \left|\sum_{x \in S} (p_x -  q_x) \right|\\
		&= \max_S \left|\sum_{x \in S} r_x \right|
\end{align*}


Since $\sum_{x \in S} r_x $ is written as
\begin{align}
	\sum_{x \in S} r_x = \sum_{\substack{x \in S\\  r_x \geq 0}} r_x + \sum_{\substack{x \in S\\ r_x < 0}} r_x,\label{eq:subsettracedist}
\end{align}
$\left|\sum_{x \in S} r_x \right|$ is maximized when $S = \{x \in U | r_x \geq 0  \}$ or $S = \{x \in U | r_x < 0  \}$.

Define $S_+ = \{x \in U | r_x \geq 0  \}$ and $S_- = \{x \in U | r_x < 0  \}$.

Now the sum of all $r_x$ is 0,
\begin{align*}
	&\sum_{x \in U} r_x = \sum_{x \in S_+} r_x + \sum_{x \in S_-} r_x = 0\\
	\therefore &\sum_{x \in S_+} r_x = - \sum_{x \in S_-} r_x.
\end{align*}

Thus
\begin{align}
	\max_S \left|\sum_{x \in S} r_x \right| = \sum_{x \in S_+} r_x = - \sum_{x \in S_-} r_x \label{eq:probtracedist}.
\end{align}

On the other hand,
\begin{align}
	D(p_x, q_x) &= \frac{1}{2} \sum_{x \in U} |p_x - q_x|\nonumber\\
		&= \frac{1}{2} \sum_{x \in U} |r_x|\nonumber\\
		&= \frac{1}{2} \sum_{x \in S_+} |r_x| + \frac{1}{2} \sum_{x \in S_-} |r_x|\nonumber\\
		&= \frac{1}{2} \sum_{x \in S_+} r_x - \frac{1}{2} \sum_{x \in S_-} r_x\nonumber\\
		&= \frac{1}{2} \sum_{x \in S_+} r_x + \frac{1}{2} \sum_{x \in S_+} r_x ~~~(\because \text{eqn} (\ref{eq:probtracedist}))\nonumber\\
		&= \sum_{x \in S_+} r_x \nonumber\\
		&= \max_S \left|\sum_{x \in S} r_x \right|.\nonumber
\end{align}

Therefore $D(p_x, q_x) = \max_S \left|\sum_{x \in S} p_x - \sum_{x \in S}q_x \right| = \max_S |p(S) - q(S)|$.

\Textbf{9.5}
From eqn (\ref{eq:subsettracedist}) and (\ref{eq:probtracedist}), maximizing $\left|\sum_{x \in S} r_x \right|$ is equivalent to
maximizing $\sum_{x \in S} r_x$.

Hence
\begin{align*}
	D(p_x, q_x) = \max_S (p(S) - q(S)) = \max_S \left(\sum_{x \in S} p_x - \sum_{x \in S}q_x \right).
\end{align*}



\Textbf{9.6}

Define $\rho = \frac{3}{4}\kb{0} + \frac{1}{4}\kb{1}$, $\sigma = \frac{2}{3} \kb{1} + \frac{1}{3}\kb{1}$.
\begin{align*}
	D(\rho, \sigma) &= \frac{1}{2} \Tr |\rho - \sigma|\\
		&= D((3/4, 1/4), (2/3, 1/3))\\
		&= \frac{1}{2} \left(\left| \frac{3}{4} - \frac{2}{3}  \right| +\left| \frac{1}{4} - \frac{1}{3} \right| \right)\\
		&= \frac{1}{2} \left( \frac{1}{12} + \frac{1}{12} \right)\\
		&= \frac{1}{12}
\end{align*}


Define $\rho = \frac{3}{4}\kb{0} + \frac{1}{4}\kb{1}$, $\sigma = \frac{2}{3} \kb{+} + \frac{1}{3}\kb{-}$.
\begin{align*}
	\kb{+} &= \frac{1}{2} (\kb{0} + \kbt{0}{1} + \kbt{1}{0} + \kb{1})\\
	\kb{-} &= \frac{1}{2} (\kb{0} - \kbt{0}{1} - \kbt{1}{0} + \kb{1})
\end{align*}

\begin{align*}
	\rho - \sigma &= \left(\frac{3}{4} - \frac{1}{2}\right) \kb{0} - \frac{1}{6} (\kbt{0}{1} + \kbt{1}{0}) + \left(\frac{1}{4} - \frac{1}{2}\right) \kb{1}\\
		&= \frac{1}{4} \kb{0} - \frac{1}{6} (\kbt{0}{1} + \kbt{1}{0}) - \frac{1}{4} \kb{1}\\
\end{align*}

\begin{align*}
	(\rho - \sigma)^\dagger (\rho - \sigma) &= \frac{1}{4^2} \kb{0} - \frac{1}{4\cdot 6} \kbt{0}{1} + \frac{1}{6^2} \kb{0} + \frac{1}{6 \cdot 4} \kbt{0}{1} - \frac{1}{4\cdot 6} \kbt{1}{0} + \frac{1}{6^2} \kb{1} + \frac{1}{4\cdot 6} \kbt{1}{0} + \frac{1}{4^2} \kb{1}\\
		&= \left(\frac{1}{4^2} + \frac{1}{6^2}\right) (\kb{0} + \kb{1})
\end{align*}


\begin{align*}
	D(\rho, \sigma) &= \frac{1}{2} \Tr |\rho - \sigma|\\
		&= \sqrt{\frac{1}{4^2} + \frac{1}{6^2}}
\end{align*}



\Textbf{9.7}

Since $\rho - \sigma$ is Hermitian, we can apply spectral decomposition.
Then $\rho - \sigma$ is written as
\begin{align*}
	\rho - \sigma &= \sum_{i=1}^k \lambda_i \kb{i} + \sum_{i=k+1}^{n} \lambda_i \kb{i}
\end{align*}
where $\lambda_i$ are  positive eigenvalues for $i = 1, \cdots, k$ and negative eigenvalues for $i = k+1, \cdots, n$.

Define $Q = \sum_{i=1}^k \lambda_i \kb{i}$ and $S = -\sum_{i=k+1}^{n} \lambda_i \kb{i}$.
Then $P$ and $S$ are positive operator. Therefore $\rho - \sigma = P - S$.

\begin{screen}
	Proof of $|\rho - \sigma| = Q + S$.
	\begin{align*}
		|\rho- \sigma| &= |Q - S|\\
			&= \sqrt{(Q-S)^\dagger (Q-S)}\\
			&= \sqrt{(Q-S)^2}\\
			&= \sqrt{Q^2 - QS - SQ + S^2}\\
			&= \sqrt{Q^2 + S^2}\\
			&= \sqrt{\sum_i \lambda_i^2 \kb{i}}\\
			&= \sum_i |\lambda_i| \kb{i}\\
			&= Q + S
	\end{align*}
\end{screen}

\Textbf{9.8}

Suppose $\sigma = \sigma_i$. Then $\sigma = \sum_i p_i \sigma_i$.
\begin{align}
	D \left( \sum_i p_i \rho_i, \sigma\right) &= D \left( \sum_i p_i \rho_i, \sum_i p_i \sigma_i\right)\\
		&\leq \sum_i p_i D(\rho_i, \sigma_i) ~~~ (\because \text{eqn}(9.50))\\
		&= \sum_i p_i D(\rho_i, \sigma).~~(\because \text{assumption}).
\end{align}

\Textbf{9.9}
\Textbf{9.10}
\Textbf{9.11}
\Textbf{9.12}

Suppose $\rho = \frac{1}{2} (I + \vec{r}\cdot \vec{\sigma})$ and $\sigma = \frac{1}{2} (I + \vec{s}\cdot \vec{\sigma})$ where $\vec{v}$ and $\vec{s}$ are real vectors s.t. $|\vec{v}|, |\vec{s}| \leq 1$.

\begin{align*}
	\mathcal{E} (\rho) = p \frac{I}{2} + (1-p) \rho, ~~~
	\mathcal{E}(\sigma) = p \frac{I}{2} + (1-p) \sigma.
\end{align*}

\begin{align*}
	D(\mathcal{E}(\rho), \mathcal{E}(\sigma)) &= \frac{1}{2} \Tr |\mathcal{E}(\rho) -  \mathcal{E}(\sigma)|\\
		&= \frac{1}{2} \Tr |(1-p)(\rho - \sigma)|\\
		&= \frac{1}{2}(1-p) \Tr |\rho - \sigma|\\
		&= (1-p) D(\rho, \sigma)\\
		&= (1-p) \frac{|\vec{r} - \vec{s}|}{2}
\end{align*}

Is this strictly contractive?

\Textbf{9.13}

Bit flip channel $E_0 = \sqrt{p} I$,  $E_1 = \sqrt{1-p}\sigma_x$.
\begin{align*}
	\mathcal{E}(\rho) &= E_0 \rho E_0^\dagger + E_1 \rho E_1^\dagger\\
		&= p \rho + (1-p) \sigma_x \rho \sigma_x.
\end{align*}

Since $\sigma_x \sigma_x \sigma_x = \sigma_x$, $\sigma_x \sigma_y \sigma_x = -\sigma_y$ and $\sigma_x \sigma_z \sigma_x = -\sigma_z$, then $\sigma_x (\vec{r} \cdot \vec{\sigma}) = r_1 \sigma_x - r_2 \sigma_y - r_3 \sigma_3$.

Thus
\begin{align*}
	D(\mathcal{E}(\rho), \mathcal{E}(\sigma)) &= \frac{1}{2} \Tr |\mathcal{E}(\rho) -  \mathcal{E}(\sigma)|\\
		&= \frac{1}{2} \Tr |p(\rho - \sigma) + (1-p) (\sigma_x \rho \sigma_x - \sigma_x \sigma \sigma_x) |\\
		&\leq \frac{1}{2}p \Tr |\rho - \sigma | + \frac{1}{2} (1-p) \Tr |\sigma_x (\rho - \sigma) \sigma_x|\\
		&= pD(\rho, \sigma) +  (1-p) D(\sigma_x \rho \sigma_x, \sigma_x \sigma \sigma_x)\\
		&= D(\rho, \sigma)~~~(\because \text{eqn}(9.21)).
\end{align*}


Suppose $\rho_0 = \frac{1}{2}(I + \vec{r}\cdot \vec{\sigma})$ is a fixed point. Then
\begin{align*}
	&\rho_0 = \mathcal{E}(\rho_0) =p \rho_0 + (1-p) \sigma_x \rho_0 \sigma_x\\
	\therefore~ &(1-p) \rho_0 - (1-p) \sigma_x \rho_0 \sigma_x = 0\\
	\therefore~ &(1-p) (\rho - \sigma_x \rho_0 \sigma_x) = 0\\
	\therefore~ &\rho_0 = \sigma_x \rho_0 \sigma_x\\
	\therefore~ &\frac{1}{2} (I + r_1 \sigma_x + r_2 \sigma_y + r_3 \sigma_z)  \frac{1}{2} (I + r_1 \sigma_x - r_2 \sigma_y - r_3 \sigma_z)
\end{align*}

Since $\{I, \sigma_x, \sigma_y, \sigma_z \}$ are linearly independent, $r_2 = -r_2$ and $r_3 = - r_3$. Thus $r_2 = r_3 = 0$.

Therefore the set of fixed points for the bit flip channel is $\{\rho~ |~ \rho = \frac{1}{2}(I + r \sigma_x), |r| \leq 1, r \in \mathds{R} \}$


\Textbf{9.14}

\begin{align*}
	F(U \rho U^\dagger, U \sigma U^\dagger) &= \Tr \sqrt{(U\rho U^\dagger)^{1/2} \sigma (U \rho U^\dagger)}\\
		&= \Tr \sqrt{U \rho^{1/2} \sigma \rho^{1/2} U^\dagger}\\
		&= \Tr (U \sqrt{\rho^{1/2} \sigma \rho^{1/2}} U^\dagger)\\
		&= \Tr (\sqrt{\rho^{1/2} \sigma \rho^{1/2}} U^\dagger U)\\
		&= \Tr \sqrt{\rho^{1/2} \sigma \rho^{1/2}}\\
		&= F(\rho, \sigma)
\end{align*}

\begin{screen}
	I think the fact $\sqrt{UAU^\dagger} = U\sqrt{A}U^\dagger$ is not restricted for positive operator.

	Suppose $A$ is a normal matrix. From spectral theorem, it is decomposed as
	\begin{align*}
		A = \sum_i a_i \kb{i}.
	\end{align*}

	Let $f$ be a function. Then
	\begin{align*}
		f(UAU^\dagger )&= f(\sum_i a_i U \kb{i} U^\dagger)\\
			&= \sum_i f(a_i) U \kb{i} U^\dagger\\
			&= U (\sum_i f(a_i) U \kb{i} U^\dagger) U^\dagger\\
			&= U f(A) U^\dagger
	\end{align*}
\end{screen}


\Textbf{9.15}
$\ket{\psi} = (U_R \otimes \sqrt{\rho}U_Q) \ket{m}$ is any fixed purification of $\rho$, and $\ket{\phi} = (V_R \otimes \sqrt{\sigma}V_Q) \ket{m}$ is purification of $\sigma$.
Suppose $\sqrt{\rho} \sqrt{\sigma} = |\sqrt{\rho} \sqrt{\sigma}| V$ is the polar decomposition of $\sqrt{\rho} \sqrt{\sigma}$. Then
%
\begin{align*}
    |\braket{\psi | \phi}| &= \left|\braket{m | \left( U_R^\dagger V_R \otimes U_Q^\dagger \sqrt{\rho} \sqrt{\sigma} V_Q \right) | m } \right|\\
        &= \left| \Tr \left((U_R^\dagger V_R)^T U_Q^\dagger \sqrt{\rho} \sqrt{\sigma} V_Q \right)\right|\\
        &= \left| \Tr \left( V_R^T U_R^* U_Q^\dagger \sqrt{\rho} \sqrt{\sigma} V_Q \right)\right|\\
        &= \left| \Tr \left( V_Q V_R^T U_R^* U_Q^\dagger \sqrt{\rho} \sqrt{\sigma} \right)\right|\\
        &= \left| \Tr \left( V_Q V_R^T U_R^* U_Q^\dagger |\sqrt{\rho} \sqrt{\sigma}| V \right)\right|\\
        &= \left| \Tr \left( V V_Q V_R^T U_R^* U_Q^\dagger |\sqrt{\rho} \sqrt{\sigma}| \right)\right|\\
        &\leq \Tr | \sqrt{\rho} \sqrt{\sigma}|\\
        &= F(\rho, \sigma)
\end{align*}

Choosing $V_Q = V^\dagger$, $V_R^T = (U_Q^* U_R^\dagger)^\dagger$ we see that equality is attained.


\Textbf{9.16}
I think eq (9.73) has a typo. $\Tr (A^\dagger B) = \braket{m | A \otimes B | m}$ should be $\Tr (A^{\textcolor{red}{T}} B) = \braket{m | A \otimes B | m}$. See errata list.

In order to show that this exercise, I will prove following two properties,
\begin{align*}
    \Tr(A) = \braket{m | (I \otimes A) | m}, ~~ (I \otimes A)\ket{ m} = (A^T \otimes I)\ket{m}
\end{align*}
where $A$ is a linear operator and $\ket{m}$ is unnormalized maximally entangled state, $\ket{m} = \sum_i \ket{ii}$.

\begin{align*}
    \braket{m |I \otimes A| m} &= \sum_{ij} \braket{ii | (I \otimes A) | jj}\\
        &= \sum_{ij} \braket{i | I | j} \braket{i | A | j}\\
        &= \sum_{ij} \delta_{ij} \braket{i | A | j}\\
        &= \sum_i \braket{i | A | i}\\
        &= \Tr (A)
\end{align*}

Suppose $A = \sum_{ij} a_{ij} \kbt{i}{j}$.
\begin{align*}
     (I \otimes A)\ket{ m} &= \left(I \otimes  \sum_{ij} a_{ij} \kbt{i}{j}\right) \sum_k \ket{kk}\\
        &= \sum_{ijk} a_{ij} \ket{k} \otimes \ket{i} \braket{j | k}\\
        &= \sum_{ijk} a_{ij} \ket{k} \otimes \ket{i} \delta_{jk}\\
        &= \sum_{ij} a_{ij} \ket{j} \otimes \ket{i}\\
        &= \sum_{ij} a_{ji} \ket{i} \otimes \ket{j}
\end{align*}

\begin{align*}
    (A^T \otimes I) \ket{m} &= \left(\sum_{ij} a_{ji} \kbt{i}{j} \otimes I\right) \sum_k \ket{kk}\\
        &= \sum_{ij} a_{ji} \ket{i} \braket{j | k} \otimes \ket{k}\\
        &= \sum_{ij} a_{ji} \ket{i} \delta_{jk} \otimes \ket{k}\\
        &= \sum_{ij} a_{ji} \ket{ij}\\
        &= (I \otimes A)\ket{ m }
\end{align*}

Thus
\begin{align*}
    \Tr (A^T B) = \Tr(BA^T) &= \braket{m | I \otimes BA^T |m}\\
    &= \braket{m | (I \otimes B)(I \otimes A^T) |m}\\
    &= \braket{m | (I \otimes B)(A \otimes I) |m}\\ 
    &= \braket{m | A \otimes B | m}.
\end{align*}

\Textbf{9.17}
If $\rho = \sigma$, then $F(\rho, \sigma) = 1$. Thus $A(\rho, \sigma) = \arccos F(\rho, \sigma) = \arccos 1 = 0$.

If $A(\rho, \sigma) = 0$, then $\arccos F(\rho, \sigma) = 0 \Rightarrow \cos (\arccos F(\rho, \sigma)) = \cos(0) \Rightarrow F(\rho, \sigma) = 1$ ($\because$ text p.411, the fifth line from bottom).


\Textbf{9.18}
For $0 \leq x \leq y \leq 1$, $\arccos(x) \geq \arccos(y)$. From $F(\mathcal{E}(\rho), \mathcal{E}(\sigma)) \geq F(\rho, \sigma)$ and $0 \leq F(\mathcal{E}(\rho), \mathcal{E}(\sigma)), F(\rho, \sigma) \leq 1$,
\begin{align*}
    \arccos F(\mathcal{E}(\rho), \mathcal{E}(\sigma)) &\geq \arccos F(\rho, \sigma)\\
    \therefore A (\mathcal{E}(\rho), \mathcal{E}(\sigma)) &\geq A(\rho, \sigma)
\end{align*}



\Textbf{9.19}
From eq (9.92)
\begin{align*}
    F \left( \sum_i p_i \rho_i, \sum_i p_i \sigma_i \right) &\geq \sum_i \sqrt{p_i p_i} F(\rho_i, \sigma_i)\\
        &= \sum_i p_i F(\rho_i, \sigma_i).
\end{align*}


\Textbf{9.20}
Suppose $\sigma_i = \sigma$. Then
\begin{align*}
    F \left( \sum_i p_i \rho_i, \sigma \right) &= F \left( \sum_i p_i \rho_i, \sum_i p_i \sigma \right) \\
        &= F \left( \sum_i p_i \rho_i, \sum_i p_i \sigma_i \right) \\
        &\geq \sum_i p_i F(\rho_i, \sigma_i)~~(\because \text{Exercise}  9.19)\\
        &= \sum_i p_i F(\rho_i, \sigma)
\end{align*}


\Textbf{9.21}
\begin{align*}
    1 - F(\ket{\psi} ,\sigma)^2 &= 1 - \braket{\psi | \sigma | \psi} ~~(\because \text{eq} (9.60))
\end{align*}


\begin{align*}
    D(\ket{\psi}, \sigma) &= \max_P \Tr (P (\rho - \sigma)) ~~(\text{where } P \text{ is projector.})\\
        &\geq \Tr \left(\kb{\psi} (\rho - \sigma)\right)\\
        &= \braket{\psi | (\kb{\psi} - \sigma) | \psi}\\
        &= 1 - \braket{\psi | \sigma | \psi}\\
        &= 1 - F(\ket{\psi}, \sigma)^2.
\end{align*}

\Textbf{9.22}
(ref: QCQI Exercise Solutions (Chapter 9) - めもめも\\ \url{http://enakai00.hatenablog.com/entry/2018/04/12/134722})

For all $\rho$, following inequality is satisfied,
\begin{align*}
        d(VU \rho U^\dagger V^\dagger, \mathcal{F} \circ \mathcal{E}(\rho))
        &\leq d(VU \rho U^\dagger V^\dagger, \mathcal{F}(U\rho U^\dagger)) + d(\mathcal{F}(U\rho U^\dagger), \mathcal{F}\circ \mathcal{E}(\rho)) \\
        &\leq d(VU \rho U^\dagger V^\dagger) + d(U\rho U^\dagger, \mathcal{E}(\rho))\\
        &\leq E(V, \mathcal{F})+ E(U, \mathcal{E}).
\end{align*}
First inequality is triangular inequality, second is contractivity of the metric\footnote{Trace distance and angle are satisfied with contractive (eq (9.35), eq (9.91)), but I don't assure that arbitrary metric satisfied with contractive. }
and third is from definition of $E$.

Above inequality is hold for all $\rho$. Thus $E(VU, \mathcal{F} \circ \mathcal{E}) \leq E(V, \mathcal{F})+ E(U, \mathcal{E})$.

\Textbf{9.23}
($\Leftarrow$)
If $\mathcal{E}(\rho_j) = \rho_j$ for all $j$ such that $p_j > 0$, then
\begin{align*}
    \bar{F} &= \sum_j p_j F(\rho_j, \mathcal{E}(\rho_j))^2 = \sum_j p_j F(\rho_j, \rho_j)^2 = \sum_j p_j 1^2 = \sum_j p_j = 1.
\end{align*}

($\Rightarrow$) Suppose $\mathcal{E}(\rho_j) \neq \rho_j$.  Then $F(\rho_j, \mathcal{E}(\rho_j)) < 1$ ($\because$  text p.411, the fifth line from bottom ).
Thus
\begin{align*}
    \bar{F} = \sum_j p_j F(\rho_j, \mathcal{E}(\rho_j))^2 < \sum_j p_j = 1.
\end{align*}
Therefore if $\bar{F} = 1$, then $\mathcal{E}(\rho_j) = \rho_j$.


\Textbf{Problem 1}
\Textbf{Problem 2}

\Textbf{Problem 3}
Theorem 5.3 of
"Theory of Quantum Error Correction for General Noise",
Emanuel Knill, Raymond Laflamme, and Lorenza Viola,
Phys. Rev. Lett. 84, 2525 – Published 13 March 2000.
arXiv:quant-ph/9604034
\url{https://arxiv.org/abs/quant-ph/9604034}



